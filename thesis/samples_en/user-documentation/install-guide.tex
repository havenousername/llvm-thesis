\section{Installation and Usage}

Depending on your needs your can either install the whole LLVM project or just clang-tidy. In the case of installing the whole project, the best way will be to create binaries by yourself as it is recommended by the LLVM documentation. 

\subsection{Install LLVM project (optional)}

There is no need in installing a complete LLVM project to use clang-tidy. As it is a standalone tool, there are a lot of ways described below how it can be used by itself or integrated into other tools like "CodeChecker". However, in some examples, there will be used clang-tidy version built from the LLVM project, so it is still advised not to skip this part of the installation. 

\subsubsection{System requirements}

The table it is shown the system requirements to install LLVM. Checkers were developed on Linux Mint 20.1 (Kernel version: 5.4.0-58-generic).

\begin{table}[H]
	\centering
	\begin{tabular}{ | m{0.25\textwidth} | m{0.25\textwidth} | m{0.25\textwidth} |}
            \hline
		\textbf{Operating System} & \textbf{Processor Architecture} & \textbf{Compiler}   \\ 
		  \hline
		Linux & x861 & gcc, clang\\
		\hline
		Linux & amd64 & gcc, clang\\
		\hline
		Linux & arm & gcc, clang\\
		\hline
		Linux & Mips & gcc, clang\\
		\hline
		Linux & PowerPC & gcc, clang\\
		\hline
		Solaris & V9 & gcc\\
		\hline
		FreeBSD & x861 & gcc, clang\\
		\hline
		FreeBSD & amd64 & gcc, clang\\
		\hline
		NetBSD & x861 & gcc, clang\\
		\hline
		NetBSD & amd64 & gcc, clang\\
		\hline
		macOS2 & PowerPC & gcc\\
		\hline
		macOS & x86 & gcc, clang\\
		\hline
		Cygwin & x86 & gcc\\
		\hline
		Windows & x86 & Visual Studio\\
		\hline
		Windows64 & x86-64 & Visual Studio\\
		\hline
	\end{tabular}
	\caption{LLVM system requirements}
	\label{tab:sys-req}
\end{table}


\subsubsection{Prerequisites}

\begin{table}[H]
    \centering
    \begin{tabular}{|m{0.2\textwidth}|m{0.2\textwidth}| m{0.2\textwidth}|}
        \hline
        \textbf{Package} & \textbf{Version} & \textbf{Notes} \\ 
        \hline
        \href{https://cmake.org/}{CMake} & >=3.13.4	& Workspace generator \\ 
        \hline
        \href{https://gcc.gnu.org/}{GCC} & >=7.1.0 & C/C++ compiler \\ 
        \hline
        \href{https://python.com}{python} & >=3.6 & Runs ".py" commands from source  \\ 
        \hline
        \href{http://savannah.gnu.org/projects/make}{GNU Make} & 3.79, 3.79.1 & Makefile build processor \\ 
        \hline
    \end{tabular}
    \caption{System Prerequisites}
    \label{tab:prerequisites}
\end{table}


\subsubsection{Building LLVM from the source}

\begin{listing}[H]
\begin{minted}{bash}
$ # on Windows --config core.autocrlf=false flag need to be added to the git clone command.
$ git clone https://github.com/llvm/llvm-project.git
$ # Cloning into 'llvm-project'...
$ # remote: Enumerating objects: 5166681, done.
$ # remote: Counting objects: 100% (2333/2333), done.
$ # remote: Compressing objects: 100% (333/333), done.
$ # Receiving objects:   0% (8261/5166681), 2.82 MiB | 195.00 KiB/s
$ cd llvm-project
$ mkdir build 
$ cd build
$ # Configure build using cmake
$ cmake \
$ -G "Ninja" \ -DLLVM_ENABLE_PROJECTS="llvm;clang;clang-tools-extra" \  DCMAKE_BUILD_TYPE=Release \
-DBUILD_SHARED_LIBS=ON \
-DLLVM_TARGETS_TO_BUILD=X86 \ -DCMAKE_EXPORT_COMPILE_COMMANDS=ON \
../llvm/
$ # -j 4 number of parallel jobs done by the ninja. Even with all 4 cores used it might take a while
$ ninja -j 4
\end{minted}
\caption{Build LLVM}
\label{code:build-llvm-usr}
\end{listing}

Unfortunately, as of now, the version of clang-tidy with EXP-45 checker is not yet uploaded to the release version of LLVM. Therefore if you have \href{https://mainstream.inf.elte.hu/}{ELTE Gitlab Mainstream account} you can use the following commands: 
\begin{listing}[H]
\begin{minted}{bash}
$ # on Windows --config core.autocrlf=false flag need to be added to the git clone command.
$ git clone --recurse-submodules https://github.com/havenousername/llvm-thesis
$ # Cloning into 'llvm-thesis'...
$ # remote: Enumerating objects: 5166681, done.
$ # remote: Counting objects: 100% (2333/2333), done.
$ # remote: Compressing objects: 100% (333/333), done.
$ # Receiving objects:   0% (8261/5166681), 2.82 MiB | 195.00 KiB/s
$ cd llvm-thesis/llvm-projecct
$ # Same steps as for the release version
\end{minted}
\caption{Build LLVM with CERT EXP-45 checker}
\label{code:build-llvm-exp45-usr}
\end{listing}


\subsection{Install clang-tidy}

\paragraph{Package installation on Linux(or WSL)/Mac OS\\}

\begin{enumerate}
    \item For Ubuntu/Debian distributions: 
        \begin{listing}[h]
        \begin{minted}{bash}
            $ sudo apt-get update -y
            $ sudo apt-get install -y clang-tidy
        \end{minted}
        \caption{Install from bash on Ubuntu/Debian}
        \label{code:install-debian}
        \end{listing}
    \item For Fedora/Centos distributions: 
        \begin{listing}[h]
        \begin{minted}{bash}
            $ dnf install clang-tools-extra
        \end{minted}
        \caption{Install from bash on Fedora/Centos}
        \label{code:install-fedora}
        \end{listing}
    \item For Mac OS:
        \begin{listing}[h]
        \begin{minted}{bash}
            $ brew install llvm
            $ ln -s "$(brew --prefix llvm)/bin/clang-tidy" "/usr/local/bin/clang-tidy"
        \end{minted}
        \caption{Install from bash on Mac OS}
        \label{code:install-fedora}
        \end{listing}
\end{enumerate}

\subsection{Clang-tidy usage}

To demonstrate how clang-tidy works, we will use main.c file with the following code:

\begin{listing}[h]
\begin{minted}{c}
#include <stdio.h>

int main() {
  printf("Here is print\n");
  // Here clang-tidy should create two warnings
  int x, y = 0;
  // Here as well
  if (x = y) {
    printf("%d", x);
  }
}
\end{minted}
\caption{Sample C program for clang-tidy diagnostics}
\label{code:sample-c-usr}
\end{listing}


The simplest way to run clang-tidy is to call it directly (if you have installed it from the package manager, if not, add an alias to ".bash\_rc"), with the \lstinline{checks=*,*} . 

\begin{listing}[h]
\begin{minted}{bash}
$ # simple usage is following:
$ # clang-tidy path/to/file -checks=*,*
$ clang-tidy main.c -checks=-*,*
\end{minted}
\caption{Run clang-tidy with all checks}
\label{code:run-checks-all}
\end{listing}

This command should result in the following output: 

\begin{figure}[H]
    \centering
    \caption{Output of \lstinline{clang-tidy path/to/file -checks=*,*}}
    \includegraphics[width=0.8\textwidth,height=200px]{images/tidy-usage/01.png}
    \label{fig:clang-tidy-usage-01}
\end{figure}

For checking the assignment-in-selection checker (or any other specific checker/checkers), the argument of the \lstinline{checks} checks should be modified the in following way: \lstinline{-checks=*,checker01,checker02}. \lstinline{-checks} accepts a certain type of Regular Expression (the string before the comma defines which checkers will be ignored by clang-tidy, everything after it will define the rules for accepted checkers) as in figure \ref{fig:clang-tidy-usage-02}.

\begin{figure}[H]
    \centering
    \caption{Output of \lstinline{clang-tidy path/to/file -checks=*,cert-assignments-in-selection}}
    \includegraphics[width=0.8\textwidth,height=150px]{images/tidy-usage/02.png}
    \label{fig:clang-tidy-usage-02}
\end{figure}

\paragraph{Create a text file from clang-tidy warning \\}

It is possible to send clang-tidy to the text output instead of the terminal. Prerequisites for Linux is \lstinline{tee} library. Here is an example of how it can be done:
\begin{listing}[h]
\begin{minted}{bash}
$ Build/bin/clang-tidy -header-filter=-* Build/main.cpp  -checks=-*,cert-assignments-in-selection,bugprone-* | tee out.log
\end{minted}
\caption{Output warnings to the text file}
\label{code:out-text}
\end{listing}

\begin{figure}[H]
    \centering
    \includegraphics[width=0.8\textwidth,height=150px]{images/tidy-usage/text-log.png}
    \label{fig:clang-tidy-usage-03}
    \caption{Text logs}
\end{figure}

\paragraph{Create an HTML file from warnings}

\href{https://github.com/austinbhale/clang-tidy-html}{clang-tidy-html} library is the perfect tool for creating HTML representation of the clang-tidy warnings. 

\begin{listing}[h]
\begin{minted}{bash}
$ # Prerequisites: python3 (https://www.python.org/)
$ pip3 install clang-html
$ # if you did not create .log file use this command: Build/bin/clang-tidy -header-filter=-* Build/main.cpp  -checks=-*,cert-assignments-in-selection,bugprone-* | tee out.log
$ pip3 install clang-html
$ python -m clang_html out.log -o index.html
\end{minted}
\caption{Install and use clang-html}
\label{code:out-html}
\end{listing}


\begin{figure}[H]
    \centering
    \caption{Text logs}
    \includegraphics[width=1\textwidth,height=150px]{images/tidy-usage/html-log.png}
    \label{fig:clang-tidy-usage-04}
\end{figure}

There are many other options for how clang-tidy could be used and configured. You can address official \href{https://clang.llvm.org/extra/clang-tidy/index.html}{documentation} for a more detailed explanation of those.